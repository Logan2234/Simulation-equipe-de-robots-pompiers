\documentclass[a4paper,8pt]{article} %style de document
\usepackage[utf8]{inputenc} %encodage des caractères
\usepackage[french]{babel} %paquet de langue français
\usepackage[T1]{fontenc} %encodage de la police
\usepackage[top=2cm,bottom=2cm,left=2cm,right=2cm]{geometry} %marges
\usepackage{graphicx} %affichage des images
\usepackage{fancyhdr}
\usepackage{verbatim}
\usepackage{hyperref}
\usepackage{appendix}
\usepackage{amsmath}
% \usepackage[french]{algorithme} %on pourrait ausi utiliser [english] pour la langue des mots-clés

\title{Rapport TPL POO}
\author{Equipe Teide: 10}
\date{} %suppression de l'affichage de la date

\begin{document}

\maketitle
\section{Choix de conception}
\subsection{Structure des sources} %différents dossiers et classes abstraites
Pour notre projet, nous avons fait le choix de rassembler en différents packages nos sources pour les organiser: 
\begin{itemize}
    \item Le package \texttt{Donnees} comporte tous les fichiers de classe des objets que l'on va manipuler pendant
    la simulation (\texttt{Robot}, \texttt{Carte}, \texttt{Case}, \texttt{Direction}\ldots).
    \item Le package \texttt{Robot}, lui même dans le package \texttt{Donnees}, comporte notre classe abstraite de \texttt{Robot} ainsi
    que toutes les classes de robots qui en hérite.
    \item Le package \texttt{Evenements} comporte tous les évènements ainsi que la classe abstraite \texttt{Evenement}
    \item Le package \texttt{Exceptions} comporte toutes les exceptions rajoutées par nos soins.
    \item Le package \texttt{Tests} comporte tous les tests créés pour tester les différentes parties du sujet. On y trouve également
    La classe \texttt{Simulation} et l'énumérateur \texttt{Test} qui liste tous nos tests (son utilité sera expliquée plus bas).
    \item Le package \texttt{Chefs} comporte la classe abstraite \texttt{Chef} ainsi que \texttt{ChefBasique} et \texttt{ChefAvance} qui mettent
    en place les deux stratégies proposées par l'énoncé.
    \item Le package \texttt{io} gère les entrées et sorties du programme. 
    On y retrouve donc la classe utilitaire finale \texttt{LecteurDonnees} construisant les \texttt{DonneesSimulation}, la classe \texttt{LecteurDonneesTextuel} (classe initialement fournie et nommée \texttt{LecteurDonnees}) et la classe utilitaire finale \texttt{Dessin} regroupant les fonctions utiles pour la fonction \texttt{draw} de la classe \texttt{Simulation}. 
    \item Enfin, le package \texttt{Autre} comporte la classe \texttt{Chemin} et la classe utilitaire finale \texttt{CalculPCC} qui s'occupe de calculer le chemin le plus rapide entre deux cases.
    
\end{itemize}

\subsection{Les exceptions}

Nous avons décidé de rajouter quelques exceptions afin de rendre notre projet un peu plus robuste. 
Ainsi, {\tt NoFireException} (resp. {\tt NoWaterException}) est soulevée lorsque qu'un robot essaie d'intéragir avec une case n'étant pas en feu (resp.\ de se remplir sans eau à proximité).
{\tt NoMoreFireException} est soulevée quand il n'y a plus de feu sur la carte (la simulation est donc terminée) tandis que {\tt NoMoreRobotsException} est lancée quand plus aucun robot n'a d'eau dans son réservoir (et que le chef n'est pas en capacité de leur ordonner de se remplir).

\subsection{Ajout de méthodes dans les objets}
\subsubsection{getLastDate dans Robot}
\subsubsection{getLitresInit dans Incendie}
\subsection{Classe Simulation}

La classe {\tt Simulation} permet de factoriser le code des fonctions utilisées dans chacun des tests, que ce soit la fonction {\tt draw}, {\tt next} ou encore {\tt restart}.
C'est cette classe qui implémente l'interface {\tt Simulable}.

\subsection{Algorithme de plus court chemin: Dijkstra}

Pour réaliser ses déplacements dans le moindre nombre de mouvements possibles, nos robots doivent trouver le plus court chemin pour atteindre le feu à éteindre. Pour cela, nous devons considérer notre mappe comme étant un graphe pondéré où tous les poids sont positifs ou nuls. Ainsi, les sommets du graphe seraient les cases de la mappe et le poids serait le temps dépensé par le robot subissant le déplacement pour aller d'une case à une autre. Nous avons donc décidé d'utiliser un des algorithmes classiques de la Recherche Opérationnelle avec une compléxité $\theta(n^2)$ : l'algorithme de Dijkstra. \\ \\
Ainsi, nous avons implémenté l'algorithme en adaptant le pseudo-code suivant : \\ \\

\begin{tabbing}
{\tt Entrées :} mappe et robot à déplacer, ainsi que la case d'arrivé.\\
{\tt Sortie : } le meilleur chemin pour réaliser le déplacement. \\ \\
dist $\leftarrow [|\infty,...,\infty|]$          matrice de même taille que la mappe \\
dist(case du robot) = 0\\
chemins $\leftarrow$          matrice de chemins vides de même taille que la mappe\\
chemins(case du robot) = case du robot \\
ouverts $\leftarrow [|0,m.n|]$          avec m et n les dimensions de la mappe\\
{\tt tant \= que} ouvert $\neq \oslash$ {\tt faire}\\
\> soit j $\in$ ouverts tel que dist.(j) soit minimal\\
\> ouverts $\leftarrow $ ouverts $\setminus$ {j}\\
\>{\tt pour} k \= voisin accessible par la case j {\tt faire} \\
\>\> d $\leftarrow$ dist.(j) + distance(j $\rightarrow$ k)\\
\> \> {\tt si} d \=< dist(k) {\tt alors} \\
\> \> \> dist.(k) $\leftarrow$ d\\
\> \> \> chemins(k) = chemins(s) $\Rightarrow$ k\\
return chemins.(case d'arrivé)
\end{tabbing}

\\ \\ \\

Le robot qui cherche à se déplacer exécute cette fonction et ajoute au simulateur le mouvement vers la case où se trouve le feu. \\
La fonction a été codée dans le fichier {\tt CalculPCC} sous forme de fonction statique, utilisée à chaque fois que le robot souhaite recharger son réservoir ou éteindre un feu.


\subsection{Les chefs pompiers}
La structure des deux chefs est commune, d'où l'utilisation d'une classe abstraite \texttt{Chef}. Typiquement, on retrouve:
\begin{enumerate}
    \item Le HashSet \texttt{occupes} qui garde tous les robots qui sont occupés (se dirige vers un incendie, l'éteint,
    se dirige vers une source d'eau ou se remplit).
    \item La méthode \texttt{strategie} qui va être la seule méthode publique: elle va boucler sur nos incendies afin de leur assigner un robot.
\end{enumerate}

\subsubsection{Chef basique}
La  classe \texttt{ChefBasique} implémente la méthode basique proposée par le sujet pour aller éteindre les incendies.
Après lecture du sujet, il restait des zones d'ombres. On a donc fait deux choix:
\begin{itemize}
    \item Les robots ne vont jamais se remplir: cet aspect sera vu dans \texttt{ChefAvance};
    \item Le concept de pas n n'est pas utilisé, on le fixe à 1 et on va lancer la fonction \texttt{gestionIncendie} systématiquement 
    et faire du traitement que s'il y a des robots non occupés.
\end{itemize}

Pour stocker l'état de nos robots et de nos incendies, nous avons créé la HashMap \texttt{incendies\_rob} 
qui va stocker en clé les incendies encore allumés et en valeur le robot qui leur est affecté. L'intérêt d'avoir accès à la valeur de robot,
permet de voir si ce robot a vidé son eau, si c'est le cas il est envoyé dans le HashSet \texttt{morts} et est retiré du HashSet \texttt{occupes}.

Le code a été factorisé avec deux méthodes privées supplémentaires:
\begin{enumerate}
    \item \texttt{gestionIncendies} qui va envoyer un robot sur un incendie qui n'est pas éteint. S'il trouve un robot pas occupé et pas mort,
    il va alors appeler la fonction \texttt{donneOrdre}. Pendant son appel, il va en profiter pour faire le tri dans les robots s'ils sont
    mal placés (pas encore mort alors que le réservoir est vide, encore occupe alors qu'il n'est plus associé à un incendie et qu'il lui reste
    de l'eau);
    \item \texttt{donneOrdre} va donner l'ordre au robot d'aller jusqu'à l'incendie et d'intervenir dessus avec les événements nécessaires.
\end{enumerate}

\subsubsection{Chef avancé}
\section{Tests effectués et résultats}
\subsection{Le fonctionnement des tests}
\subsubsection{Lancer les tests}
Les différents tests du package \texttt{Tests} sont là pour tester les différentes parties du sujet.
\begin{itemize}
    \item \texttt{TestSimulation} s'assure que l'on peut lancer une simulation.
    \item \texttt{TestScenarios} lance les scénarios proposés dans la partie 2.2 du sujet.
    \item \texttt{TestDijkstra} a pour but de vérifier que l'algorithme de Dijkstra fonctionne. Sur la carte sujet, 
    on sélectionne le second robot, qui est le robot à pattes et on l'envoie sur le premier incendie (celui en bas à gauche).
    On veut s'assurer qu'il contourne l'eau et passe bien par les rochers et ne fasse pas de détours.
    \item \texttt{TestBasique} teste le chef pompier avec l'algorithme élémentaire du sujet.
    \item \texttt{TestAvance} teste le chef pompier avec l'algorithme plus poussé fourni par le sujet.
\end{itemize}

\subsubsection{Makefile}
Le Makefile et les tests rajoutés sont lancés sur la map du sujet par défaut.
Pour compiler lancer les tests: 
\begin{itemize}
    \item \texttt{testSimulation} pour compiler \texttt{TestSimulation} puis \texttt{exeSimulation} pour exécuter;
    \item \texttt{testScenarios} pour compiler \texttt{TestScenarios} puis \texttt{exeScenarios} pour exécuter;
    \item \texttt{testDijkstra} pour compiler \texttt{TestDijkstra} puis \texttt{exeDijkstra} pour exécuter;
    \item \texttt{testBasique} pour compiler \texttt{TestBasique} puis \texttt{exeBasique} pour exécuter;
    \item \texttt{testAvance} pour compiler \texttt{TestAvance} puis \texttt{exeAvance} pour exécuter;
\end{itemize}


\subsection{Résultats obtenus}
\subsection{Pistes d'amélioration}

\end{document}