\documentclass[a4paper,8pt]{article} %style de document
\usepackage[utf8]{inputenc} %encodage des caractères
\usepackage[french]{babel} %paquet de langue français
\usepackage[T1]{fontenc} %encodage de la police
\usepackage{graphicx} %affichage des images
\usepackage{fancyhdr}
\usepackage{verbatim}
\usepackage{hyperref}
\usepackage{appendix}
\usepackage{amsmath}
\usepackage[french]{algorithme} %on pourrait ausi utiliser [english] pour la langue des mots-clés

\title{Rapport TPL POO}
\author{Equipe Teide : 10}
\date{} %suppression de l'affichage de la date

\begin{document}
\maketitle
\section{Choix de conception}
\subsection{Structure des sources} %différents dossiers et classes abstraites
Pour notre projet, nous avons fait le choix de rassembler en différents packages nos sources pour les organiser : 
\begin{itemize}
    \item Le package \texttt{Donnees} comporte tous les fichiers de classe des objets que l'on va manipuler pendant
    la simulation (\texttt{Robot}, \texttt{Carte}, \texttt{Case}, \texttt{Direction} ...).
    \item Le package \texttt{Robot}, lui même dans le package \texttt{Donnees}, comporte notre classe abstraite de \texttt{Robot} ainsi
    que toutes les classes de robots.
    \item Le package \texttt{Evenements} comporte tous les évènements sélectionnables ainsi que la classe abstraite \texttt{Evenement}\\
    \item Le package \texttt{Exceptions} comporte toutes les exceptions rajoutées par nos soins. Elles seront expliquées plus bas.
    \item Le package \texttt{Tests} comporte tous les tests créés pour tester les différentes parties du sujet. On y trouve également
    La classe \texttt{Simulation} et l'énumérateur \texttt{Test} qui liste tous nos tests (utile pour le Makefile).
    \item Le package \texttt{Chefs} comporte la classe abstraite \texttt{Chef} ainsi que \texttt{ChefBasique} et \texttt{ChefAvance} qui mettent
    en place les deux stratégies proposées par l'énoncé.
    \item Enfin, le package \texttt{Autre} comporte les classes \texttt{Chemin} et \texttt{CalculPCC} qui s'occupe de gérer les chemin et de 
    calculer le plus rapide.
    
\end{itemize}

\subsection{Les exceptions}
\subsection{Ajout de méthodes dans les objets}
\subsubsection{getLastDate dans Robot}
\subsubsection{getLitresInit dans Incendie}
\subsection{Classe Simulation}
\subsection{Algorithme de plus court chemin : Dijkstra}
\subsection{Les chefs pompiers : structure et algorithme}
\section{Tests effectués et résultats}
\subsection{Le fonctionnement des tests}
\subsubsection{Makefile et enum Test}
\subsubsection{Lancer les tests}
\subsection{Résultats obtenus}
\subsection{Pistes d'amélioration}

\end{document}